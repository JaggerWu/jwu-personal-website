\cvsection{Research Experience}
\begin{cventries}
  \cventry
    {Master Thesis}
    {Deep Contact: Accelerating Rigid Simulation with Convolutional Networks}
    {Copenhagen, Denmark}
    {Feb. 2018 - Exp.Sep. 2018}
    {
      \begin{cvitems}
        \item {\url{https://github.com/JaggerWu/Deep-Contact}}
	\item Final grade, 10/12(The second best score). 
	\item SPH(smoothed-particle hydrodynamics with poly6 kernel) method for transforming the simulate state to accessful data for traing. CNN for training data.
      \end{cvitems}
    }
 \cventry
    {Implementing the algorithms mentioned in popular Computer Vision paper}
    {Advanced Topics in Image Analysis}
    {Copenhagen, Denmark}
    {Nov. 2017 - Jan. 2018}
    {
      \begin{cvitems}
        \item {\url{https://github.com/JaggerWu/Advance-Topic-in-Image-Analysis}}
        \item {The whole project is consist of three main parts, including solutions for three basic problems in Image Analysis, Segmentation, Objects Recognition, 3D reconstruct}
        \item {The segmentation part mainly use mean-shift cluster.}
        \item {The Objects Recognition part mainly using CNN (AlexNet) with pre-processing the original images.}
        \item {The 3D part mainly focus on Fundamental Matrix estimation, including choosing correct match points by RANSAC and optimize matrix estimation.}
      \end{cvitems}
    }
  %\cventry
  %  {Free topics project in DIKU}
  %  {Weed detection based on CNN}
  %  {Copenhagen, Denmark}
  %  {Sep. 2017 - Nov. 2017}
  %  {
  %    \begin{cvitems}
  %      \item {\url{https://github.com/JaggerWu/weed-recognization}}
  %      \item {Trained CNN model for classification insteaf of classical color and texture analysis. }
  %      \item {Analyzed the performance of CNN on Image classification with pre-processing and without pre-processing }
  %    \end{cvitems}
  %  }
  \cventry
    {The final project for AI course in DTU}
    {Artificial Intelligence with Multiple Agents}
    {Copenhagen, Denmark}
    {Jan. 2017 - Jun. 2017}
    {
      \begin{cvitems}
        \item {\url{https://github.com/JaggerWu/AI_Prog_Proj}}
        \item {You can find the final solution examples in YouTuBe.
        \begin{itemize}
            \item \url{https://youtu.be/kr02LBVFr-Y}
            \item \url{https://youtu.be/6SLdyZth9mY}
        \end{itemize}}
        \item {The project was about designing algorithm which could play most pushing/pulling box game.(The game was to push or pull boxes to the correct position by single or multiple agents)}
        \item {Designed the algorithm based on basic greedy, breadth-first search and depth-first search algorithm}
        \item {Tried Reinforcement Learning, but the final result was not the best.}
      \end{cvitems}
    }
  \cventry
    {Personal interesting project}
    {Photometric Stereo}
    {Copenhagen, Denmark}
    {Jan. 2017 - Feb. 2017}
    {
      \begin{cvitems}
        \item {\url{https://github.com/JaggerWu/Photometric-Stereo}}
        \item {Reconstruct 3D image based on the same image in different lights}
      \end{cvitems} 
    }
  \cventry
    {Project for Natural Language Processing}
    {Sentiment Analysis}
    {Copenhagen, Denmark}
    {Jan. 2017 - Apr. 2017}
    {
      \begin{cvitems}
        \item {\url{https://github.com/JaggerWu/Sentiment-Analysis}}
        \item {Research two methods applied in Sentiment Analysis, Bag of Words and Word2Vec}
        \item {Applied different machine learning algorithm on classify sentiment words, including SVM, Random Forest, Naive Bayes classifier.}
      \end{cvitems} 
    }
  \cventry
    {Digital image processing curriculum design}
    {Extraction of  the moving object contour from video}
    {Lanzhou, Denmark}
    {Sep. 2014 - Dec. 2014}
    {
      \begin{cvitems}
        \item {Extract numerous frames of images in the video for simple image processing and comparing.}
        \item {Designed statistical algorithm and getting the final profile.}
      \end{cvitems} 
    }
  \cventry
    {Funded by Digital Signal Processing Lab in Lanzhou University}
    {Intelligent piano spectrum identification system}
    {Lanzhou, China}
    {Jul. 2012 - Jun. 2013}
    {
      \begin{cvitems}
        \item {Writing intelligent piano spectrum identification system which can listen to the song and turn it into harmonica tone.}
        \item {Skillfully used digital to analog conversion, the discrete Furier transform, fast Furier transform and other related knowledge.}
        \item {The related algorithms are completed by my own writing. Taking the piano song for sampling, using the discrete Furier transform analysis in time domain into frequency domain analysis, frequency of each note is obtained to determine all the notes.; adding different harmonics to change its timbre.}
      \end{cvitems}
    }
  \cventry
    {The final project for C++ language course}
    {English study software}
    {Lanzhou, China}
    {Jun. 2013, Aug. 2013}
    {
      \begin{cvitems}
        \item {The software can realize the dictionary, repeater and vocabulary testing function.}
        \item {Initially, it was a just local service without GUI, which can run locally. I finished the GUI, and finally it was sold to a local primary school to help them improve English teaching.}
      \end{cvitems}
      %\begin{cvsubentries}
      %  \cvsubentry{}{KNOX(Solution for Enterprise Mobile Security) Penetration Testing}{Sep. 2013}{}
      %  \cvsubentry{}{Smart TV Penetration Testing}{Mar. 2011 - Oct. 2011}{}
      %\end{cvsubentries}
    }
  \cventry
    {Funded by Innovation and Entrepreneurship Program from Gansu Province Government}
    {Leaf Species Recognition based on Neural Network}
    {Lanzhou, China}
    {May. 2013 - Dec. 2013}
    {
      \begin{cvitems}
        \item {Pre-processing for each image was to extract each leaf scan from complicated background.}
        \item {The main learning process is to extract SIFT feature from the leaf scan images and then using 5-layers Neural Network to classify these feature data.}
      \end{cvitems}
    }
\end{cventries}
